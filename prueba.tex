\documentclass[11pt,]{article}
\usepackage[margin=1in]{geometry}
\newcommand*{\authorfont}{\fontfamily{phv}\selectfont}
\usepackage[]{mathpazo}
\usepackage{abstract}
\renewcommand{\abstractname}{}    % clear the title
\renewcommand{\absnamepos}{empty} % originally center
\newcommand{\blankline}{\quad\pagebreak[2]}

\providecommand{\tightlist}{%
  \setlength{\itemsep}{0pt}\setlength{\parskip}{0pt}} 
\usepackage{longtable,booktabs}

\usepackage{parskip}
\usepackage{titlesec}
\titlespacing\section{0pt}{12pt plus 4pt minus 2pt}{6pt plus 2pt minus 2pt}
\titlespacing\subsection{0pt}{12pt plus 4pt minus 2pt}{6pt plus 2pt minus 2pt}

\titleformat*{\subsubsection}{\normalsize\itshape}

\usepackage{titling}
\setlength{\droptitle}{-.25cm}

%\setlength{\parindent}{0pt}
%\setlength{\parskip}{6pt plus 2pt minus 1pt}
%\setlength{\emergencystretch}{3em}  % prevent overfull lines 

\usepackage[T1]{fontenc}
\usepackage[utf8]{inputenc}

\usepackage{fancyhdr}
\pagestyle{fancy}
\usepackage{lastpage}
\renewcommand{\headrulewidth}{0.3pt}
\renewcommand{\footrulewidth}{0.0pt} 
\lhead{}
\chead{}
\rhead{\footnotesize POSC 0000: A Class with an R Markdown
Syllabus -- Fall 2016}
\lfoot{}
\cfoot{\small \thepage/\pageref*{LastPage}}
\rfoot{}

\fancypagestyle{firststyle}
{
\renewcommand{\headrulewidth}{0pt}%
   \fancyhf{}
   \fancyfoot[C]{\small \thepage/\pageref*{LastPage}}
}

%\def\labelitemi{--}
%\usepackage{enumitem}
%\setitemize[0]{leftmargin=25pt}
%\setenumerate[0]{leftmargin=25pt}




\makeatletter
\@ifpackageloaded{hyperref}{}{%
\ifxetex
  \usepackage[setpagesize=false, % page size defined by xetex
              unicode=false, % unicode breaks when used with xetex
              xetex]{hyperref}
\else
  \usepackage[unicode=true]{hyperref}
\fi
}
\@ifpackageloaded{color}{
    \PassOptionsToPackage{usenames,dvipsnames}{color}
}{%
    \usepackage[usenames,dvipsnames]{color}
}
\makeatother
\hypersetup{breaklinks=true,
            bookmarks=true,
            pdfauthor={ ()},
             pdfkeywords = {},  
            pdftitle={POSC 0000: A Class with an R Markdown Syllabus},
            colorlinks=true,
            citecolor=blue,
            urlcolor=blue,
            linkcolor=magenta,
            pdfborder={0 0 0}}
\urlstyle{same}  % don't use monospace font for urls


\setcounter{secnumdepth}{0}

\usepackage{longtable}




\usepackage{setspace}

\title{POSC 0000: A Class with an R Markdown Syllabus}
\author{Steven V. Miller}
\date{Fall 2016}


\begin{document}  

		\maketitle
		
	
		\thispagestyle{firststyle}

%	\thispagestyle{empty}


	\noindent \begin{tabular*}{\textwidth}{ @{\extracolsep{\fill}} lr @{\extracolsep{\fill}}}


E-mail: \texttt{\href{mailto:svmille@clemson.edu}{\nolinkurl{svmille@clemson.edu}}} & Web: \href{http://svmiller.com/teaching}{\tt svmiller.com/teaching}\\
Office Hours: T 4:30-5:30 p.m. \& Th 12:00 - 1:00 p.m.
(Zoom)  &  Class Hours: TTh 1:30-2:45 p.m.\\
Office: TBD  & Class Room: Hugh Gillis Hall 116\\
	&  \\
	\hline
	\end{tabular*}
	
\vspace{2mm}
	


\begin{center}\rule{0.5\linewidth}{0.5pt}\end{center}

\hypertarget{catalog-course-description}{%
\section{Catalog Course Description}\label{catalog-course-description}}

Comparative analysis of different kinds of political systems; their
political institutions, processes and policies; the environments in
which they occur and their consequences.

\hypertarget{detailed-course-description}{%
\section{Detailed Course
Description}\label{detailed-course-description}}

This introductory course to comparative politics will help students to
better understand political processes across nations. In my view,
college students not only should be aware of the most important
questions about the world today, but also need to look beyond the
American political system and learn how political institutions succeed
or fail elsewhere. Because this is an introductory course, we will
survey a wide range of topics including states and political regimes,
political mobilization, political culture, political participation,
elections and voting, interest groups, political parties, parliaments,
executives, economic development, and globalization.

\emph{This is an Online Education course. All assignments and exams will
be conducted over the Internet. Students are responsible for their own
access to the Internet and computer resources.}

\hypertarget{expectations-and-activities}{%
\section{Expectations and
Activities}\label{expectations-and-activities}}

Success in this class will depend upon your ability to: (1) think
critically; (2) read and write University-level English prose; (3)
develop an ability to understand and systematically apply the basics of
research design; (4) work independently and in group when needed.

I expect students to:

\begin{enumerate}
\def\labelenumi{\arabic{enumi})}
\tightlist
\item
  read this syllabus carefully,
\item
  log on to the class web site a minimum of four times each week,
\item
  read all assigned materials,
\item
  watch lectures,
\item
  submit assignments and exams on time,
\end{enumerate}

DO NOT enroll in an online course if you know that you are going to be
away from your Internet access for more than 5 or 6 days during the
length of the course. Unless you have an extenuating circumstance, you
have to submit all the assignments by the deadline.

Please note that students are responsible for their own Internet access
and computing resources. A loss of connectivity is not an excuse for
late assignments. Some Internet service providers (ISPs) are notorious
for inferior, unreliable service. In previous semesters, students have
lost Internet connectivity in the middle of exams. Students who wait
until the last possible moment to submit an assignment also run the risk
of an unanticipated service disruption that prevents timely submission.

\hypertarget{faculty-webpage-and-mysjsu-communication}{%
\section{Faculty Webpage and MYSJSU
Communication}\label{faculty-webpage-and-mysjsu-communication}}

I will post announcements on Canvas on a regular basis. They will appear
on your dashboard when you log in and/or will be sent to you directly
through your preferred method of notification from Canvas. Please make
sure to check them regularly, as they will contain any important
information about upcoming projects or class concerns.

In this course we will use the CONVERSATIONS feature on the help corner
(located in navigation links) to send email for private messages. You
can either check your messages in the Canvas system or set your
notifications to your preferred method of contact. Please check your
messages regularly.

I receive many emails from students everyday, and I try to respond to
all of them in a timely manner. An email is a formal communication
between you and your professor, and it thus should be addressed
properly. For my students, I am Dr.~Bejar or Dr.~Bejar-Lopez. Please
keep this in mind when you send me an email. \textbf{I will not respond
to emails that are not properly addressed.}

\hypertarget{creating-an-environment-of-mutual-respect}{%
\section{Creating an Environment of Mutual
Respect}\label{creating-an-environment-of-mutual-respect}}

This class is a partnership between you, your classmates and your
professor. Together, we will build a supportive, respective, and
productive environment to learn and to explore challenging questions
about International Political Economy. Building this kind of environment
requires mutual respect.

What do I expect from you, to create an environment of mutual respect? I
expect you to complete the readings and watch all the posted lectures. I
also expect professional behavior in the class and to remain engaged
throughout the semester. Lack of interest or engagement is likely to be
reflected in your grade.

What can you expect from me? You can expect me to be tirelessly
enthusiastic and to work hard for you, both in this semester and in
future semesters if needed. I encourage all of you to stop by my virtual
office hours, even if you don't have a question and just would like to
chat about the class, life after SJSU or life in general. You can reach
me best via email at
\href{mailto:sergio.bejar@sjsu.edu}{\nolinkurl{sergio.bejar@sjsu.edu}}.

\hypertarget{department-of-political-science-learning-outcomes}{%
\section{Department of Political Science Learning
Outcomes}\label{department-of-political-science-learning-outcomes}}

\begin{enumerate}
\def\labelenumi{\arabic{enumi}.}
\tightlist
\item
  Breadth: Students should possess a broad knowledge of the theory and
  methods of the various branches of the discipline.
\item
  Application and Disciplinary Methods: Students should be able to
  formulate research questions, engage in systematic literature searches
  using primary and secondary sources, evaluate research studies, and
  critically analyze and interpret influential political texts. Students
  should be able to apply these techniques to identify, understand, and
  analyze domestic and international political issues and organizations.
\item
  Communication Skills: Students should master basic competencies in
  oral and written communication skills and be able to apply these
  skills in the context of political science. This means communicating
  effectively about politics and/or public administration, public
  policy, and law.
\item
  Citizenship: Students should acquire an understanding of the role of
  the citizen in local, state, national, and global contexts and
  appreciate the importance of lifelong participation in political
  processes.
\end{enumerate}

\hypertarget{course-learning-outcomes}{%
\section{Course Learning Outcomes}\label{course-learning-outcomes}}

This class satisfies the D2 general education requirement (Comparative
Systems, Cultures and Environments). Upon successful completion of this
course, students will be able to:

1.Place contemporary developments in cultural, historical, environmental
and spatial contexts; 2. Identify the dynamics of ethnic, cultural,
gender/sexual, age-based, class, regional, national, transnational, and
global identities and the similarities, differences, linkages, and
interactions between them; 3. Evaluate social science information, draw
on different points of view, and formulate applications to appropriate
to contemporary social issues; 4. Compare and contrast two or more
ethnic groups, cultures, regions, nations, or social systems.

Assessment of these outcomes will be measured as follows:

\begin{enumerate}
\def\labelenumi{\arabic{enumi}.}
\tightlist
\item
  GELO 1: Discussion boards, Twitter engagement, exams, group projects.
\item
  GELO 2: Quizzes, Twitter engagement, group projects.
\item
  GELO 3: Discussion boards, exams.
\item
  GELO 4: Quizzes, discussion boards, Twitter engagement.
\end{enumerate}

\textbf{See below for a detailed description of each of the
aforementioned assignments and their requirements}

\hypertarget{e-mail-policy}{%
\section{E-mail Policy}\label{e-mail-policy}}

I am usually quick to respond to student e-mails. However, student
e-mails tend to do several things that try my patience. I have a new
policy, effective Spring 2021, that outlines why I will not respond to
certain e-mails students send. Multiple rationales follow.

\begin{enumerate}
\def\labelenumi{\arabic{enumi}.}
\tightlist
\item
  The student could answer his/her own inquiry by reading the syllabus.
\item
  The student missed assignments or exams. I do not need to know the
  exact reason for a missed assignment or exam. Students with excusable
  reasons are responsible for giving me a note \emph{in hard copy} that
  documents the reason for the missed class.
\item
  The student wants to know what topics have been covered in class. The
  answer is always ``you missed what was on the syllabus.''
\item
  The student is protesting a grade without reference to specific points
  of objection. These e-mails tend to be expressive utility on the part
  of the student and do not require a response from me. Students
  interested in improving their knowledge of material should see me
  during office hours.
\item
  The student is requesting an extension on an assignment for which the
  syllabus already established the deadline. The answer is always
  ``no''.
\item
  The student is
  \href{https://www.math.uh.edu/~tomforde/GradeGrubbing.html}{``grade
  grubbing''} or asking to round up a grade. The answer is always
  ``no''.
\item
  The student is asking for an extra credit opportunity, a request that
  amounts to more grading for the professor. The answer is ``no''.
\item
  The student emails during the weekend.
\end{enumerate}

\hypertarget{course-readings}{%
\section{Course Readings}\label{course-readings}}

\href{https://www.pearson.com/us/higher-education/product/Samuels-Comparative-Politics-RENTAL-EDITION-2nd-Edition/9780134562674.html}{David
Samuels, Comparative Politics, Second Edition, Pearson, 2018.}

Students will also read a variety of newspaper and magazine articles.
The course's Canvas page will have the links to those additional
readings.

\hypertarget{assignment-weights-and-due-dates}{%
\section{Assignment Weights and Due
Dates}\label{assignment-weights-and-due-dates}}

\begin{longtable}[]{@{}ccc@{}}
\toprule()
\textbf{Assignment} & \textbf{Weight} & \\
\midrule()
\endhead
Discussion Boards & 5 x 5\% each = 25\% & \\
Group Projects & 3 x 10\% each = 30\% & \\
Exams & 2 x 17.5\% each = 35\% & \\
Quizes & 5 X 2\% each = 10\% & \\
\bottomrule()
\end{longtable}

\hypertarget{description-of-assignments}{%
\section{Description of Assignments}\label{description-of-assignments}}

\textbf{Discussion Boards:} Students will participate in 5 graded
discussion boards (there will be other non-graded boards). In each
graded discussion you will be expected to make 3 posts: your initial
post (minimum 200 words) and replies to at least two of your classmates
(minimum 100 words each).

The nature of these posts varies. But you should expect to get questions
on the documnetaries and short videos that you are required to watch as
well as on the newspaper and magazine articles that are part of your
coursework. Your postings should be well written and clearly address the
issues being discussed. I expect each writing assignment to have: (1) A
clear introduction that addresses directly the question posed by the
instructor; (2) A body of factual examples that support your thesis;
these examples may be drawn from either the assigned readings or
footnoted sources researched independently by the student; (3)
Appropriate source citations; plagiarized threads will be penalized. (4)
A succinct concluding paragraph. Your responses must be posted by the
deadline specified on Canvas. .

If I have some concerns or comments about your thread, I will post a
response. My comments are intended to help you improve your threads. If
you respond to my comments, you may earn additional points. In order to
earn a perfect score, you generally have to post an excellent thread the
first time around and by the assigned deadline. In addition, you must
respond to the threads of at least two other students 24 hrs. after the
deadline.

\textbf{Group Projects:} Once the final roster of the class is
available, I will randomly assign you to a working group (or team). As a
team, you will craft two (2) short essays. The topics are below. Each
brief should be 1,000 words long (max).

\begin{enumerate}
\def\labelenumi{\arabic{enumi}.}
\item
  \emph{Online Game}: You will meet your classmates virtually to play
  the \href{https://3rdworldfarmer.org/}{3rd World Farmer} game (fun!).
  This activity should not take more than 20-30 minutes. You are
  responsible to organize the online meeting on Zoom or any other
  platform that allows you to communicate with your classmates. As a
  group, you will then write a short report (500 words) about your
  experience playing the game. How did you make decisions? Why? What are
  the main lessons you learned from playing the game?
\item
  \emph{Country Profile:} Each group will be assigned a country
  different from the United States. You will present a country profile
  consisting of a short narrative and key economic and and political.
  The narrative should focus on the country's modern evolution -- the
  most salient political parties (PRI if studying Mexico, for example),
  social or political cleavages (agrarian elites versus Evo Morales in
  Bolivia for example) and major current events (The Olympics and their
  backlash in Japan).
\item
  \emph{Current Events:} With your team, you will present the most
  salient political or economic event preoccupying the public, or
  government, currently. This might be a regional dispute, an
  anti-terrorist operation, an indigenous group's protest, or an
  economic crisis. You can draw from class resources, though you may
  need to read ahead, or outside of class, in order to best grapple with
  the analysis.
\end{enumerate}

\textbf{Quizzes:} There will be 5 quizzes. Each of them will ask
questions about the video lecture presentation(s) of the different
modules. Success in these quizzes is simple: watch the lectures, take
notes, pay attention and review your notes before the quizz. See course
calendar below for due dates. \emph{Late quizzes --even a second late-
will receive a zero.}

\textbf{Exams:} Students will take two(2) mid-term exams. These exams
will not be cummulative and are likely to include a combination of
multiple choice, short answer and essay questions. See course calendar
below for dates.

Both exams will be administered on Canvas. They will be open-notes and
open-book exams, but you will not be allowed to collaborate with other
students in completing them. Both exams will be timed. \emph{Late exams
--even a second late- will receive a zero.}

\hypertarget{policy-on-late-work}{%
\section{Policy on Late Work}\label{policy-on-late-work}}

\textbf{Discussion board (initial posts):} initial posts can be
submitted late but will incur a 25\% penalty for each started 24-hour
period (starting at 5:01pm on the day they are due). This means that you
have 72 hrs. before your response receives automatically a 0.

\textbf{Discussion board (replies to classmates):} no late replies to
classmates are allowed. The thread will close at 5:01pm on the day
replies is due and no further submissions will be allowed.

\textbf{Group Policy Briefs:} your briefs will be penalized 25\% for
each started 24-hour period (starting at 5:01pm).

\textbf{Exams and quizzes:} exam and quiz make-ups are only given in
cases of medical or family emergencies, in accordance with the
university's policy on excused absences. In these cases, you MUST notify
me before the exam and proper documentation must be provided.

\hypertarget{grading-scale}{%
\section{Grading Scale}\label{grading-scale}}

\begin{longtable}[]{@{}lc@{}}
\toprule()
Grade & Percentage \\
\midrule()
\endhead
A plus & 98-100\% \\
A & 94-97.9\% \\
A minus & 90-93.9\% \\
B plus & 87-89.9\% \\
B & 84-86.9\% \\
B minus & 80-83.9\% \\
C plus & 77-79.9\% \\
C & 74-76.9\% \\
C minus & 70-73.9\% \\
D plus & 67-69.9\% \\
D & 64-66.9\% \\
D minus & 60-63.9\% \\
F & 0-59.9\% \\
\bottomrule()
\end{longtable}

\hypertarget{public-sharing-of-instructor-material}{%
\section{Public Sharing of Instructor
Material}\label{public-sharing-of-instructor-material}}

Students are prohibited from distributing, sharing, or posting class
lectures, slides, exams, or any other instructional materials. Materials
created by the instructor for the course (syllabi, lectures and lecture
notes, presentations, exams, etc.) are copyrighted by the instructor.
\href{https://www.sjsu.edu/senate/docs/S12-7.pdf}{University policy
S12-7} is in place to protect the privacy of students in the course, as
well as to maintain academic integrity through reducing the instances of
cheating. Students who record, distribute, or post these materials will
be referred to the Student Conduct and Ethical Development office.
Unauthorized recording may violate university and state law. It is the
responsibility of students that require special accommodations or
assistive technology due to a disability to notify the instructor.

\hypertarget{academic-dishonesty}{%
\section{Academic Dishonesty}\label{academic-dishonesty}}

Students who are suspected of cheating during an exam/quiz/assignment
will be referred to the Student Conduct and Ethical Development office
and depending on the severity of the conduct, will receive an F in the
course. Grade Forgiveness does not apply to courses for which the
original grade was the result of a finding of academic dishonesty.

\hypertarget{course-schedule-subject-to-change-with-fair-notice}{%
\section{Course Schedule (Subject to Change with Fair
Notice)}\label{course-schedule-subject-to-change-with-fair-notice}}

\begin{longtable}[]{@{}
  >{\centering\arraybackslash}p{(\columnwidth - 6\tabcolsep) * \real{0.2188}}
  >{\centering\arraybackslash}p{(\columnwidth - 6\tabcolsep) * \real{0.2812}}
  >{\centering\arraybackslash}p{(\columnwidth - 6\tabcolsep) * \real{0.2500}}
  >{\centering\arraybackslash}p{(\columnwidth - 6\tabcolsep) * \real{0.2500}}@{}}
\toprule()
\begin{minipage}[b]{\linewidth}\centering
\textbf{Module}
\end{minipage} & \begin{minipage}[b]{\linewidth}\centering
\textbf{Dates}
\end{minipage} & \begin{minipage}[b]{\linewidth}\centering
\textbf{Topics}
\end{minipage} & \begin{minipage}[b]{\linewidth}\centering
\textbf{Readings and Assignments}
\end{minipage} \\
\midrule()
\endhead
1 & Aug 18 - Aug 20 & \textbf{Introduction} & \\
& & & 1. Syllabus \\
& & & 2. Watch introductory video (Canvas) \\
& & & \textbf{Discussion board \#1 due Aug 21st (5 pm)} \\
& & & \textbf{Syllabus Quiz due Aug 21st (5 pm)} \\
& & & \\
2 & August 23- August 27 & \textbf{What is Comparative Politics?} & \\
& & & 1. Read Samuels, Chapter 1 \\
& & & 2. Listen
\href{https://www.rstreet.org/2020/10/05/podcast-what-can-we-learn-from-other-nations-about-pernicious-polarization-in-the-united-states/}{Podcast} \\
& & & \textbf{Discussion Board \#1 due August 27th (5 pm)} \\
& & & \\
3 & August 30 - September 10 & \textbf{Theories of State Formation} & \\
& & & 1. Read Samuels, Chapter 2 \\
& & & 2. Read Samuels, Chapter 10 \\
& & & 3. Read
\href{https://www.devex.com/news/tackling-the-problems-of-fragile-states-in-africa-90212}{Saldinger} \\
& & & \textbf{Respond to Classmates' Posts by September 30th (5pm)} \\
& & & \textbf{Quiz \#2 due September 3rd (5pm)} \\
& & & \textbf{Discussion Board \#2 due September 10th (5pm)} \\
& & & \\
4 & February 22 - March 5 & \textbf{Political Regimes: Democracy and
Autocracy} & \\
& & & 1. Read Dahl (See Canvas) \\
& & & 2. Read Samuels, Chapter 4. \\
& & & 3. Podcast
\href{https://www.npr.org/2018/01/22/579670528/how-democracies-die-authors-say-trump-is-a-symptom-of-deeper-problems}{How
Democracies Die} \\
& & & 4. PBS Documentary ``Commanding Heights'' \\
& & & \textbf{Respond to Classmates' Posts by September 13th, (5pm)} \\
& & & \textbf{Group Project \#1 due February 24th (5pm)} \\
& & & \textbf{Quiz \#3 due February 26th (5pm)} \\
& & & \textbf{Discussion Board \#3 due March 5th (5pm)} \\
& & & \\
5 & March 8 - March 19 & \textbf{Democracy, Religion and Economic
Development} & \\
& & & 1. Read Bueno de Mesquita (see Canvas) \\
& & & 2. Read Goldstone (see Canvas) \\
& & & \textbf{Respond to Classmates' Posts by March, 8th (5pm)} \\
& & & \textbf{Discussion Board \#4 due March 26th (5pm)} \\
& & & \\
6 & March 22 - March 26 & \textbf{Exam \#1} & \\
& & & \\
& March 29 - April 2 & \textbf{Spring Recess} & \\
& & & \\
7 & April 5 - April 16 & \textbf{Political Institutions 1: Parties,
Elections and Participation} & \\
& & & 1. Read Samuels, Chapter 9 \\
& & & 2. Read Auyero (see Canvas) \\
& & & 3. Read
\href{https://www.opendemocracy.net/en/breaking-fresh-evidence-hungary-vote-rigging-raises-concerns-fraud-european-elections/}{Open
Democracy} \\
& & & 4. Read
\href{https://cnnphilippines.com/news/2019/5/13/massive-vote-buying-pnp.html}{Vote
Buying} \\
& & & \textbf{Discussion Board \#4 due April 9th (5pm)} \\
& & & \textbf{Respond to Classmates' Posts by April 11th, (5pm)} \\
& & & \textbf{Group Project \#2 due April 14th (5 pm)} \\
& & & \\
7 & April 19 - April 30 & \textbf{Political Institutions 2:
Parliamentary/Presidential and Federal/Unitarian Systems} & \\
& & & 1.Read Samuels, Chapter 3 \\
& & & 2. Read
\href{https://www.brookings.edu/blog/fixgov/2020/06/08/how-the-constitutions-federalist-framework-is-being-tested-by-covid-19/}{Selin} \\
& & & \textbf{Quiz \#4 due April 22nd (5pm)} \\
& & & \textbf{Discussion Board \#5 due April 30th (5pm)} \\
& & & \\
9 & May 3 - May 15 & \textbf{Welfare Systems and Redistribution} & \\
& & & 1. Read Samuels, Chapter 12 \\
& & & 2. Read Samuels, Chapter 13 \\
& & & \textbf{Respond to Classmates' Posts by May 3rd (5pm)} \\
& & & \textbf{Quiz \#5 due May 10th (5pm)} \\
& & & \textbf{Group Project \#3 due May 15th (5pm)} \\
& & & \\
11 & May 19 & \textbf{Online Exam Posted, due May 21st (5pm)} & \\
\bottomrule()
\end{longtable}




\end{document}

\makeatletter
\def\@maketitle{%
  \newpage
%  \null
%  \vskip 2em%
%  \begin{center}%
  \let \footnote \thanks
    {\fontsize{18}{20}\selectfont\raggedright  \setlength{\parindent}{0pt} \@title \par}%
}
%\fi
\makeatother